% !TEX root = semexp-thesis.tex

% include with relative paths
\usepackage{import}
\newcommand{\thimport}[2][.]{%
	\IfFileExists{#1/#2/../#2}{%
		\subimport{#1/#2}{../#2}%
	}{%
		\input{#1/#2}%
	}
}


% TEMPLATE PATCHES

%% increase width of page numbers in table of contents (to match 3 instead of 2 digits)
%\makeatletter
%\renewcommand*{\@pnumwidth}{2em} % default: 1.55em
%\makeatother

% make pdf bookmark labels of chapters bold
\usepackage{bookmark}
\makeatletter
\bookmarksetup{
	open,
	addtohook={%
		\ifnum\bookmarkget{level}=0 %
			\if@mainmatter
				\bookmarksetup{bold}%
			\fi
		\fi
	},
}
\makeatother

% hide draft info
\makeatletter
\AtBeginDocument{
	\sbox{\swa@draftPageLine}{}
}
\makeatother

% patch date format
\usepackage[style=iso]{datetime2}

% convenient use of slashes with word wrapping (ported from ngerman)
\useshorthands*{"}
\defineshorthand{"/}{\slash\hspace{0pt}}

% labels in figure captions point to beginning of figure
\usepackage[all]{hypcap}

% declaration of authorship:
% - patch date format manually (datetime2 doesn't work here??)
\makeatletter
\expandafter\patchcmd{\endstatement}{\datedate}{\two@digits{\thedateyear}-\two@digits{\thedatemonth}-\two@digits{\thedateday}}{\typeout{patch 1 successful}}{\typeout{patch 1 failed}}
\makeatother
% - remove article before date
\makeatletter
\expandafter\patchcmd{\endstatement}{\@location, den }{\@location, }{\typeout{patch 2 successful}}{\typeout{patch 2 failed}}
\makeatother

% customize chapter preamble
\setkomafont{dictumtext}{\itshape\small}
\setkomafont{dictumauthor}{\normalfont}
\renewcommand*\dictumwidth{\linewidth}
\renewcommand*\dictumauthorformat[1]{--- #1\par}
\renewcommand*\dictumrule{}

% better asterism
\renewcommand{\asterism}{\smash{%
	\raisebox{-.5ex}{%
		\setlength{\tabcolsep}{.2pt}%
		*}}}

% subparagraphs: italic instead of bold
%\let\originalsubparagraph
%\renewcommand\subparagraph[1]{\originalsubparagraph{\emph{#1}}}
\addtokomafont{subparagraph}{\normalfont\itshape}

% paragraph in caption
\newcommand\capparagraph[1]{%
	\bold{#1}\hspace{1em}\ignorespaces
}

% shared list of figures and tables
% CREDITS: https://tex.stackexchange.com/a/120651/221054
\makeatletter
\renewcommand*{\ext@table}{lof}
\let\c@table\c@figure
\let\ftype@table\ftype@figure
\let\listoffiguresandtables\listoffigures
\renewcaptionname{english}{\listfigurename}{List of Figures and Tables}
\makeatother


% COMMANDS

% landscape pages
\usepackage{afterpage}
\usepackage{pdflscape}
\usepackage{environ}
\newif\ifpreviewing
\ifdraft{%
	% just convenience for some PDF readers
	\previewingtrue  % Comment out this line to revert to standard behavior; restore landscape pages to A4
}{}
\makeatletter
\newbox\landscapefigure@saved@page
\def\afterpagebody#1{%
	\afterpage{
		\begin{landscape}
			\ifpreviewing%  % CREDITS: https://tex.stackexchange.com/a/719759/221054
				\AtBeginShipoutNext{%
					\setbox\landscapefigure@saved@page=\hbox{\resizebox{!}{0.707\height}{\box\AtBeginShipoutBox}}
					\setbox\AtBeginShipoutBox=\hbox{\box\landscapefigure@saved@page}
					\pdfpageheight=210mm
					\pdfpagewidth=148.5mm
					\hoffset=-0.293in
					\voffset=-0.293in
				}%
			\fi%
			\begin{figure}
				#1
			\end{figure}
		\end{landscape}
	}%
}
\makeatother
\NewEnviron{landscapefigure}{%
	\expandafter\afterpagebody\expandafter{\BODY}%
}
\let\oldfigure\figure
\let\endoldfigure\endfigure

\makeatletter
\newcommand{\DefaultFigureOptions}{}
\begingroup
	\edef\@tempa{\csname fps@figure\endcsname}%
	\global\let\DefaultFigureOptions\@tempa
\endgroup
\makeatother
\makeatother

% add new figure Z option for landscape page
\BeforeBeginEnvironment{document}{%
	\RenewDocumentEnvironment{figure}{ O{tbp} }{%
		% hack: should use \DefaultFigureOptions here, but expansion issue
		\IfStrEq{#1}{Z}{%
			\landscapefigure
		}{%
			\oldfigure[#1]
		}%
	}{%
		\IfStrEq{#1}{Z}{%
			\endlandscapefigure
		}{%
			\endoldfigure
		}%
	}
}

% float barriers for controlling figure placements
% ALSO used as a workarond for a KNOWN LIMITATION of landscapefigure: will be delayed more than regular figures
\usepackage{placeins}


% Linebreaks within camelCaseWords
% CREDITS: https://tex.stackexchange.com/a/423540/221054
%\ExplSyntaxOn
%\NewDocumentCommand{\camelCaseHyphenated}{m}{%
%	\camelcasetext:n { #1 }%
%}
%\cs_new_protected:Nn \camelcasetext:n {
%	\tl_map_inline:nn { #1 }
%		{
%		% if the current char is uppercase, add a discretionary hyphen
%		\str_if_eq:eeT { ##1 } { \str_uppercase:n { ##1 } }
%		{ \- }
%		##1
%	}
%}
%\ExplSyntaxOff
% TODO: does not work with \textquotesingle, does not break within single words
% workaround instead:
\hyphenation{Se-man-tic-Search-Re-sult-Rank-ing}
\hyphenation{does-Not-Un-der-stand}

% line breaks anywhere in long sequences (e.g., as part of certain URLs)
\usepackage{seqsplit}

% Semantic markup
\newcommand\bold[1]{\textbf{#1}}
\renewcommand\code[1]{%
	\texttt{#1}}
	%\texttt{\camelCaseHyphenated{#1}}}
	%\textsf{#1}} % This looks more like Smalltalk
\newcommand\hardcode[1]{%
	\texttt{#1}}
\newcommand\name[1]{\textsc{#1}}
\newenvironment{multicode}{%
	\renewcommand\t{\null\qquad}%
	\newcommand\n{\newline}%
	\begin{quote}%
		\ttfamily%
		%\sffamily% % This looks more like Smalltalk
		\setlength{\parindent}{0pt}%
		\raggedright%
	}{\end{quote}}
\newcommand\printit[1]{%
	\emph{$ \to $ #1}%
}

% Manual linebreak arrows in code
\usepackage{MnSymbol}
\newcommand{\multicodeforcebreak}{%
	\\\phantom{....}\llap{\raisebox{0.2ex}[0ex][0ex]{\ensuremath{\rcurvearrowse\hspace*{2pt}}}}%
}

% boxes
\newlength{\globalparindent}
\setlength{\globalparindent}{\parindent}
\usepackage[most]{tcolorbox}
\tcbset{
	smallbox/.style={
		%title	= {\refstepcounter{exa}example~\theexa.},
		before upper	= {\parindent=\globalparindent},
		},
	largebox/.style={
		%title	= {\refstepcounter{sum}summary~\theexa.},
		before upper	= {\parindent=\globalparindent},
		},
}
% todo refactor: automatically provide paragraph/section here
\newtcolorbox{smallbox}{
	smallbox,
	breakable,
	enhanced,
	before skip=\bigskipamount,
	after skip=\bigskipamount,
}
\newtcolorbox{largebox}{
	largebox,
	breakable,
	enhanced,
	before skip=2\bigskipamount,
	after skip=\bigskipamount,
}
\usepackage{needspace}
% optional argument: number of lines to hold together before page break
\newenvironment{example}[1][3]{%
	\ifnum#1>0
		\newdimen\bigskipmax
		\bigskipmax=\dimexpr
			\bigskipamount + \gluestretch\bigskipamount % maximum is required to avoid being outsmarted by glues
		\relax
		\needspace{\dimexpr
			\bigskipmax % before skip
			+ \globalparindent % before upper
			+ \baselineskip*#1 % content
			+ \globalparindent % after upper
		\relax}
	\fi%
	\smallbox
	\paragraph{Example.}
	\setlist[1]{leftmargin=2em}
}{
	\endsmallbox%
}
\newenvironment{summary}{% TODO: rename to chaptersummary
	\largebox
	\section*{Chapter Summary}\vspace{-1em}
	\setlist[1]{leftmargin=2em}
}{%
	\endlargebox%
}
\newenvironment{genericbox}[1]{%
	\smallbox
	\paragraph{#1.}
	\setlist[1]{leftmargin=2em}
}{
	\endsmallbox%
}


% enumitem
% ... have them look nicer
\AtBeginDocument{%
	\setlist[description]{
		% copied over from swathesis
		noitemsep,font=\normalfont\scshape,leftmargin=\parindent,topsep=.5\baselineskip,partopsep=.5\baselineskip,
		listparindent=1em, % visible hanging indent for new paragraphs inside items
	}
}
% add noextralabelsep option (only insert regular space after label)
\makeatletter
\newif\ifenit@noextralabelsep
\enitkv@key{}{noextralabelsep}[true]{%
	\enit@sepfrommarginfalse
	\enitkv@setkeys{enumitem}{style=sameline} % consistent spacing in labels and text -- but unfortunately not for labelsep, which we fix below
	\enit@calcset\labelsep\tw@{0em} % eliminate static labelsep
	\global\enit@noextralabelseptrue
}
% introduce dynamic space after item instead
\let\originalitem\item
\RenewDocumentCommand{\item}{o}{%
	\IfNoValueTF{#1}{%
		\originalitem%
	}{%
		\originalitem[#1]%
		\ifenit@noextralabelsep%
			{\enit@format\ }%
			\ignorespaces%
		\fi%
	}%
}
\let\origenddescription\enddescription
\renewcommand{\enddescription}{%
	\global\enit@noextralabelsepfalse%
	\origenddescription%
}
\makeatother



% Automatic reference labels (\cref)
\usepackage[nameinlink]{cleveref}
\crefalias{subappendix}{section}

% ensure footnote references include chapter number
%\labelformat{footnote}{\thechapter.#1}
% In fact, this is more complicated because we want different formats for \cref and \footref.
% CREDITS: https://tex.stackexchange.com/a/726697/221054 and https://tex.stackexchange.com/a/351559/221054
% (1) Define separate labels + references for footnotes
\makeatletter
\newcommand{\footnotelabel}[1]% #1 = unmodified label name
{\bgroup
	\edef\@currentlabelname{\thechapter}%
	\label{#1}%
\egroup}

\newcommand{\footnoteref}[1]{\hyperlink{\getrefbykeydefault{#1}{anchor}{Doc Start}}%
	{footnote \nameref*{#1}.\ref*{#1}}} % simulates cleveref + nameinlink
\makeatother

% (2) Patch \label and \cref for footnotes
% (2.1) Allow checking the type of a label
\usepackage[hyperref,counter]{zref} % Using the counter mechanism behind `nameref`

\makeatletter
\AtBeginDocument{%
	\let\latex@@label\label%

	\renewcommand{\label}[1]{%
		\zref@label{#1}%
		\latex@@label{#1}%
	}
	% Get the underlying counter type
	\newcommand{\extractlabelcounter}[1]{%
		\zref@ifrefundefined{#1}{%
		???????}{%
		\zref@extract{#1}{counter}%
		}%
	}
	% Get the anchor name for hyperref or nameref -> has a `*` inside if it is unnumbered
	\newcommand{\extractlabelanchor}[1]{%
		\zref@ifrefundefined{#1}{%
		???????}{%
		\zref@extract{#1}{anchor}%
		}%
	}
}

% Check if there's a `*` inside of the anchor name
\ExplSyntaxOn
\cs_new:Npn \checkifnumbered#1#2#3{%
  \tl_set:Nx \l_tmpa_tl {\extractlabelanchor{#1}}
  \tl_if_in:NnTF \l_tmpa_tl {*} {#2} {#3}
}
\ExplSyntaxOff
\makeatother

\newcommand{\checklabeltype}[4]{%
	\ifnum0=\pdfstrcmp{\extractlabelcounter{#1}}{#2}
	#3%
	\else
	#4%
	\fi
}

% (2.2) Do the patches
\RequirePackage{xstring}
\AtBeginDocument{%
	\let\oldlabel\label
	\renewcommand\label[1]{%
		% if #1 begins with "fn:", use \footnotelabel, else use \oldlabel
		\IfBeginWith{#1}{fn:}{%
			\let\newlabel\label%
			\let\label\oldlabel%
			\footnotelabel{#1}%
			\let\label\newlabel%
		}{\oldlabel{#1}}}
	\let\oldcref\cref
	\renewcommand\cref[1]{%
	    \checklabeltype{#1}{footnote}{\footnoteref{#1}}{\oldcref{#1}}}
}


% absolute page counter
\usepackage{zref-abspage}


% SI numbers
\usepackage{siunitx}
% no space for percentages
\DeclareSIUnit[quantity-product = ]\percent{\char`\%}
% dollar units
\newcommand{\USD}[1]{\${\num{#1}}}
\DeclareSIUnit{\cent}{\textcent}

% Redefine \qty and \num to use text mode unless in math mode (https://tex.stackexchange.com/a/726242/221054)
\usepackage{letltxmacro}
\LetLtxMacro\origqty\qty
\renewcommand\qty[3][]{%
  \ifmmode
    \origqty[#1]{#2}{#3}%
  \else
    \origqty[mode=text,#1]{#2}{#3}%
  \fi
}
\LetLtxMacro\orignum\num
\renewcommand\num[2][]{%
  \ifmmode
    \orignum[#1]{#2}%
  \else
    \orignum[mode=text,#1]{#2}%
  \fi
}


% biblatex: separator between multiple citations: change to semicolon only for \cites
\DeclareMultiCiteCommand{\cites}[\mkbibbrackets]{\cite}{\addsemicolon\space}


% yet another table package (but a pretty nice one)
\usepackage{tabularray}
\UseTblrLibrary{siunitx}
% for subfloats in tblrs
\UseTblrLibrary{counter,varwidth}

% circled numbers
\usepackage{circledsteps}
\newcommand{\CircledInverted}[2][]{%
    \CircledParamOpts{fill color=black, inner color=white, outer color=none, #1}{1}{#2}%
}


% rainbow text (based on the gradient-text package)
% CREDITS: https://tex.stackexchange.com/a/701906/221054 (edited)
\makeatletter
\ExplSyntaxOn
\clist_new:N\l_gtext_First_clist
\clist_new:N\l_gtext_Last_clist
\int_new:N\l_gtext_MaxIndex_int
\int_new:N\l_gtext_Ratio_int
\newcommand{\gr@dient}[8]{
	\int_set:Nn\l_gtext_MaxIndex_int{\int_eval:n{\str_count:n{#1}}}
	\int_step_inline:nnn{1}{\l_gtext_MaxIndex_int}{
		\exp_args:Ne\str_if_eq:nnTF{\str_item:Nn{#1}{##1}}{~}{}{
			\int_set:Nn\l_gtext_Ratio_int{\int_eval:n{\l_gtext_Ratio_int+1}}
		}
		\color_select:nn{#8}{
			\int_mod:nn{\int_eval:n{(\int_use:N\l_gtext_Ratio_int*#5+(\l_gtext_MaxIndex_int-##1)*#2)/\l_gtext_MaxIndex_int} + 240*100}{240},
			\int_eval:n{(\int_use:N\l_gtext_Ratio_int*#6+(\l_gtext_MaxIndex_int-##1)*#3)/\l_gtext_MaxIndex_int},
			\int_eval:n{(\int_use:N\l_gtext_Ratio_int*#7+(\l_gtext_MaxIndex_int-##1)*#4)/\l_gtext_MaxIndex_int}
		}\str_item:Nn{#1}{##1}
	}
}

\NewDocumentCommand\gradient{mmmm}{{
  \clist_set:Nn\l_gtext_First_clist {#3}
  \clist_set:Nn\l_gtext_Last_clist {#4}
  \gr@dient{#2}
  {\clist_item:Nn\l_gtext_First_clist{1}}
  {\clist_item:Nn\l_gtext_First_clist{2}}
  {\clist_item:Nn\l_gtext_First_clist{3}}
  {\clist_item:Nn\l_gtext_Last_clist{1}}
  {\clist_item:Nn\l_gtext_Last_clist{2}}
  {\clist_item:Nn\l_gtext_Last_clist{3}}
  {#1}
}}
\ExplSyntaxOff

\NewDocumentCommand\gradientRGB{mmm}{
  \gradient{RGB}{#1}{#2}{#3}
}

\NewDocumentCommand\gradientHSB{mmm}{
  \gradient{HSB}{#1}{#2}{#3}
}
\makeatother


% Automatically insert placeholders for missing figures
\ifdraft{%
	\usepackage{commands/robustfigures}
}{}


% debug patching
%\tracingpatches
