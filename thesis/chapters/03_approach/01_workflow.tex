% !TEX root = ../../semexp-thesis.tex

\section{The Augmented Exploratory Programming Workflow}
\label{sec:approach/workflow}

\begin{figure}
	\centering
	% todo: consider only one arrow color OR make them look more different (colors)
	\includegraphics[width=\textwidth]{01_workflow/workflow.png}
	\caption[Our model of an \emph{augmented exploratory programming workflow}.]{
		Our model of an augmented exploratory programming workflow.
		Programmers can exchange conceptual artifacts with a \emph{semantic exploratory programming system} (\bold{\textcolor{red}{red}}) through high-level semantic interfaces.
		Based on the shared artifacts, the programming system continues the research process and suggests new artifacts to the programmer.
	}
	\label{fig:approach/workflow/workflow}
\end{figure}

We want to address the challenges of the traditional exploratory programming workflow which are caused by the limited support of exploratory programming systems and manifest as semantic distances, information overload, and interruptions in programmers.
To this end, we propose an \emph{augmented exploratory programming workflow} which describes collaborations between programmers and systems at higher abstraction levels and introduces the notion of \emph{semantic exploratory programming systems} which integrate semantic technologies to support programmers at conceptual steps in their research process~(\cref{fig:approach/workflow/workflow}).

In the augmented exploratory programming workflow, programmers can provide semantic context from their research process such as questions and plans to the exploratory programming system.
The system builds on this context to ``think along'' and continue the research process on its own:
it attempts to conduct next likely exploratory steps such as planning and executing experiments, deducing results, or answering questions and shares its work to the programmer at different granularities.
For example, a programmer could provide a high-level question to the system and receive a list of suggested experiments to execute, or the programmer could perform their own experiments and receive an automated summary of deduced results from the system.
Thus, programmers can either \emph{delegate} tasks to the system to avoid interruptions or \emph{cooperate} with it to benefit from the capabilities of semantic technologies---such as processing high quantities of information---and \emph{augment} their workflow with additional context.

To support this workflow, we propose \emph{semantic exploratory programming systems} that build upon traditional exploratory programming systems and combine them with semantic technologies to make the semantic context at higher abstraction levels of the exploratory research process accessible.
For example, this allows exploratory programming systems to interpret questions and plans, contextualize and analyze experiments, or summarize results and answer questions.
Based on this, semantic exploratory programming systems can replicate previous human steps of the research process and suggest possible continuations.

To access and contribute to the exploratory research process of human beings, semantic exploratory programming systems require new \emph{semantic interfaces} through that programmers can provide contextual artifacts as inputs and retrieve semantic suggestions and answers as outputs.
Concretely, we propose three types of semantic interface mechanisms:

\begin{description}
	\item[Anticipated experiments:]
	The semantic exploratory programming system observes the \emph{experiments} that the programmer executes through traditional interfaces.
	Based on these observations, it attempts to reconstruct their underlying plans and uses them for anticipating and suggesting further experiments to the programmer.

	\item[Semantic inputs:]
	The semantic exploratory programming system reinterprets existing or introduces new interfaces through that programmers can express their current \emph{plans}.
	For example, this involves reading quick notes of programmers, observing the setup of experiments before the programmer executes them, or---hypothetically---also listening to programmers thinking out aloud during their work.
	Based on these inputs, the system can develop a more precise image of the programmer's plans and provide more relevant suggestions to them.

	Further, the system can also attempt to reconstruct the overarching \emph{question} of the programmer based on the provided plans.
	This serves as a base for anticipating further plans and developing suggestions for them even before programmers express these plans to the system.

	Finally, the system can offer new interfaces through that programmers can explicitly express high-level questions.
	The system can use these to refine existing and new plans and provide more contextualized suggestions.

	\item[Semantic outputs:]
	Based on conceptualized ideas and questions, the semantic exploratory programming system can provide suggestions to programmers through new output interfaces on different abstraction levels.
	It can suggest low-level \emph{experiments} or execute them on its own and automatically deduce and summarize \emph{results}.
	If the original question of the programmer is available, the system can also consolidate and contextualize results to provide an \emph{answer} to this question.
\end{description}

Internally, semantic technologies are used to reconstruct plans and questions of programmers, make suggestions, and deduce results and answers.
\Cref{cha:suggestions} describes how text embeddings and semantic retrieval are employed to anticipate plans and experiments, and \cref{cha:agent} explains how systems can use generative LLMs to interpret questions, generate experiments, and deduce answers.
