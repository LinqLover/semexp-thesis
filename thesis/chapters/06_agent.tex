% !TEX root = ../semexp-thesis.tex

\chapter{Building an Exploratory Programming Agent with GPT-4o}
\label{cha:agent}

In this chapter, we describe our implementation of semantic object interfaces through a conversational autonomous agent.
In our prototype, we use OpenAI's LLM GPT-4o\footnote{Used in the version \code{gpt-4o-2024-05-13}.}, which ranks among the state-of-the-art models for our required cabilities such as problem solving and code writing at the time of writing~\cite{openai2024gpt4}.

We implement the agent's policies through prompt engineering and map the system interface to a set of functions that the LLM can call.%
\footnote{We provide the full conversation including all system prompts and function calls for the example from \cref{fig:application/text/text} in the GitHub repository (\url{https://github.com/hpi-swa-lab/SemanticSqueak/blob/11637e5/assets/Text.conversation}).} % TODO: make sure this figure exists!
Finally, we optimize the agent's performance for generating semantic code completions.

% todo: somewhere provide full prompt/conversation(s) (appendix?)

\thimport{01_prompts}
\thimport{02_interfaces}
\thimport{03_completions}

\begin{summary}
	We have implemented an exploratory programming agent for semantic object interfaces by \emph{prompting} a GPT-4o model with a set of behavioral policies and initial research steps, connecting it to the system through \emph{functions} for executing experiments, and pre-generating a statically structured prompt to efficiently generate code completions.
\end{summary}
