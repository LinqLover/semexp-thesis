% !TEX root = ../semexp-thesis.tex

\chapter{Background}
\label{cha:background}

In this chapter, we introduce the theoretical foundations of the \emph{exploratory programming} practice and provide a model of the \emph{exploratory programming workflow}.
We describe how \emph{exploratory programming systems} such as Squeak"/Smalltalk support this practice through different interaction mechanisms and tools and identify existing challenges related to the \emph{semantic immediacy} and \emph{information overload} that current exploratory programming systems pose on the workflow of exploratory programmers.
We explain two \emph{semantic technologies}---\emph{semantic retrieval} and \emph{generative large language models}---that leverage AI and ML methods to process and generate text based on their abstract meaning, and briefly sketch how these technologies could be used to meet the challenges of current exploratory programming systems and augment the exploratory programming workflow.

\thimport{01_exp}
\thimport{02_expsys}
\thimport{03_challenges}
\thimport{04_semtec}

\begin{summary}
	In the \emph{exploratory programming workflow}, programmers ask questions about software systems and answer them by incrementally decomposing abstractions and conducting experiments.
	\emph{Exploratory programming systems} support this research process by providing interfaces to the software system through low- to intermediate-level (domain-specific or task-specific) tools.
	However, traditional systems cannnot understand the context and intentions of programmers, and when programmers translate conceptual questions to technical interfaces manually, they are distracted, overwhelmed, and find only limited answers.
	\emph{Semantic technologies} provide new AI-based opportunities for processing and synthetizing information and could allow for the construction of semantic exploratory programming systems to augment the exploratory programming workflow.
\end{summary}
