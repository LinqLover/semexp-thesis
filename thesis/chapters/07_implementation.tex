% !TEX root = ../semexp-thesis.tex

\chapter{Integrating the Semantic Workspace into Squeak}
\label{cha:implementation}

In this chapter, we describe how we integrate our prototypical implementation of the semantic workspace with the required semantic technologies and the Squeak"/Smalltalk exploratory programming system.

To make semantic technologies accessible in Squeak, we use our \semtex framework, which provides a domain model for semantic search and text generation, integrates different language models from the OpenAI API, and offers fundamental tooling for prototyping semantic applications.
We provide a detailed presentation of the framework in \cref{apx:semtex}.

In the following, we briefly address particular implementation considerations for each semantic tool in the semantic workspace---semantic suggestions, semantic completions, and semantic conversations.

% TODO in general for screenshots/class diagrams: mention anywhere fictitious elements?

\thimport{01_suggestions}
\thimport{02_completions}
\thimport{03_conversations}

\begin{summary}
	We have implemented the three tools of the semantic workspace in Squeak by using our \semtex framework for accessing semantic technologies.
	We have described several design decisions and implementation details for integrating suggestions, completions, and conversations into the programming interface of Squeak.
\end{summary}
