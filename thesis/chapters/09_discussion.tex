% !TEX root = ../semexp-thesis.tex

\chapter{Discussion}
\label{cha:discussion}

In this chapter, we discuss the opportunities and challenges of semantic interfaces for augmenting exploratory programming by reporting on our experience with our prototype of the semantic workspace.
To this end, we examine the potential of semantic technologies for our applications in the semantic exploration kernel and evaluate their performance.
We then discuss the impact of semantic interfaces on the experience of exploratory programmers.
Finally, we address ethical considerations regarding the use of semantic technologies.

In \cref{apx:recommendations}, we provide distilled recommendations for other tool developers who plan to integrate semantic technologies into programming systems.

\thimport{01_feasibility}
\thimport{02_performance}
\thimport{03_experience}
\thimport{04_ethics}

\begin{summary}
	Semantic interfaces show strong promise for improving the programming experience in our experiments by streamlining the exploratory research process.
	However, programmers could miss opportunities for learning and decision-making when overusing high-level support tools, and they might not trust semantic tools that fail too often and do not explain their answers.
	To improve the accuracy and correctness of semantic tools, tool developers should invest in better training and prompting of language models, particularly for conceptual understanding of source code and exploratory programming practices.
	Similarly, high response times, monetary costs, and environmental and social impacts of current LLMs prevent programmers from using semantic completions and conversations easily and frequently, so further work is required to tune prompts, optimize models, and ideally scale them down for sustainable on-device usage.
\end{summary}
