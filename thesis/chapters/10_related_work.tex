% !TEX root = ../semexp-thesis.tex

\chapter{Related Work}
\label{cha:related_work}

In this chapter, we provide an overview of related work in the areas of suggestion mechanisms for programming systems and high-level programming interfaces.
This complements our introduction of exploratory programming practices, tools, and semantic technologies presented in \cref{cha:background}.

Our concept of a semantic exploratory programming system to that a programmer can delegate tasks shares similarities to a pair-programming setup~\cite{beck2000extreme}, where a navigator gives directions to a driver, the driver executes them, and the navigator reviews the results.
Several approaches have been proposed to mimic the role of the driver through programming tools.
Still, to our knowledge, the semantic workspace is the first work that enables conceptual collaboration of the programmer and an intelligent programming system with a variable degree of automation versus augmentation.

\thimport{01_suggestions}
\thimport{02_interfaces}

\begin{summary}
	Prior suggestion tools recommend identifiers, methods, or generated code snippets that programmers can apply to their program or use for further exploration.
	Existing high-level programming interfaces provide conceptual, often natural-language interfaces through that programmers can automate smaller search and programming tasks to facilitate programming activities such as refactoring, prototyping, and debugging.
\end{summary}
