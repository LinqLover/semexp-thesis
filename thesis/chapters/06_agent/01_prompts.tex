% !TEX root = ../../semexp-thesis.tex

\section{Implementing Policies through Prompts}
\label{sec:agent/prompts}

We define a set of \emph{policies} that define the behavior and the communication styles of the exploratory agent:

\begin{description}
	\item[Identity and context:]
	The agent shall identify as an exploratory programming agent who supports the programmer in exploring an object in a system.
	In conversation mode, the \emph{assistant} shall also identify as the focused object itself, allowing users to address the object in the second person within the chat and thus increasing the experienced semantic immediacy%
	\footnote{This follows the conversation style of message sends which are usually named in imperative or interrogative speech (such as the cascade-terminating message \code{\#yourself} in Smalltalk).}%
	.
	\item[Traits for problem solving and tool usage:]
	The agent shall adopt a structured problem-solving approach by explicating ideas and steps in advance.
	It shall check the results of experiments and iterate as necessary before providing a final answer to the programmer.
	When referring to object protocols, the agent shall automatically browse the implementation or senders of messages whose behavior is unclear, and it shall automatically test message sends by executing them before suggesting them to the programmer.
	\item[Output format:]
	The agent shall provide brief and concise answers and avoid long sentences unless requested otherwise.
	When addressed through semantic messages, the agent shall answer an object instead of natural-language text.
	Objects can be fetched from the system or be created by the agent; in particular, primitive values (such as numbers or booleans) or dynamically structured JSON objects are preferred.
\end{description}

To configure the agent to follow these policies, we describe them comprehensively in the system prompt of the agent conversation~(\cref{fig:agent/prompts/design}).
While we iterated our prompt design several times, we did not conduct a systematic, evaluation-based approach to prompt engineering~\cite{zheng2023judging}; yet our prototypical agent already showed promising results.

\paragraph*{Bootstrapping the exploration}
We initialize the internal conversation of the agent with a sequence of pre-generated messages that demonstrate the intended behavior of the agent.
These pre-generated messages feature inner monologue and (resolved) function calls by the assistant.
By this, they illustrate the agent's steps for getting a first overview of the object.
For example, our default pre-generated conversation prefix shows how the agent retrieves a textual representation of the provided object and enumerates its instance variables~(see the ``bootstrapping the exploration'' region in \cref{fig:agent/prompts/design}).

Additionally, we inject \emph{hardcoded semantic context} about the object into the pre-generated conversation prefix through system messages in natural language.
This context includes information about the role or particular messages of an object.
For example, for the class \code{Context}, which represents a stack frame in a debugger~(see \cref{fig:application/system/debugger}), we provide a brief situational explanation and point the agent to relevant protocols for querying the entire debugger stack.

\begin{figure}
	\centering
	\begin{threeparttable}
		\centering
		{\footnotesize
		\thimport{01_prompts/prompts}}
	\end{threeparttable}
	\caption[Schematic prompt design for conversations of the exploratory programming agent with the user and the system.]{
		Schematic prompt design for conversations of the exploratory programming agent with the user and the system.
		We translate policies for the identity, strategies, and output formats of the agent are translated to detailed instructions.
		Through pre-generated assistant messages and tool calls, we stimulate the agent to engage in inner monologue and several experiments.
	}
	\label{fig:agent/prompts/design}
\end{figure}

This pre-generated conversation prefix serves several purposes:
first, it provides a broad context about the object to the agent within the first invocation, which serves as the base for further experiments by the agent or maybe contains the necessary information already.
Second, the first likely actions of the agent are anticipated, improving the average response time of the agent.
Third, the prefix ``stimulates'' the agent toward an exploratory mindset, engagement in inner monologue, and eagerness to perform several experiments.
Thus, pre-generated conversation prefixes present a hybrid of zero-shot chain-of-thought prompting~\cite{kojima2022large}, few-shot prompting~\cite{brown2020language}, and retrieval-augmented generation~\cite{lewis2020retrieval}.
