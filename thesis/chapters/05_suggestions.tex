% !TEX root = ../semexp-thesis.tex

\chapter{Suggesting Experiments with Semantic Retrieval}
\label{cha:suggestions}

In this chapter, we describe how the suggestion engine creates different types of experiments to augment the exploratory research process of programmers.
We provide an overview of the different types of artifacts and strategies in the suggestion engine.
We explain how it implements different strategies for searching artifacts by using semantic retrieval, compare different ranking approaches for sorting and filtering suggestions, and describe how the suggestion engine incorporates the exploratory programming agent for suggesting code expressions.

\thimport{01_overview}
\thimport{02_search}
\thimport{03_ranking}
\thimport{04_generation}

\begin{summary}
	In this chapter, we have described how the suggestion engine anticipates experiments of programmers by capturing the context and actions of programmers as \emph{artifacts} and defining \emph{strategies} that create further artifacts based on the former.
	We have described different strategies to search the system for similar artifacts based on document embeddings and the TF-IDF metric as well as for \emph{correlated} artifacts based on frequent parts of similar artifacts.
	We have discussed different objectives and methods for \emph{ranking} artifacts.
	To optimize the suggestion of semantic completions, we have proposed a \emph{two-stage generation approach} based on the change rate of user edits.
\end{summary}
