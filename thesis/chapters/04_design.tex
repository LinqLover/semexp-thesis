% !TEX root = ../semexp-thesis.tex

\chapter{The Semantic Exploration Kernel}
\label{cha:design}

% todo: update earlier pointers (introduction)
In this chapter, we describe the design of our \emph{semantic exploration kernel}, which connects the interfaces of the different semantic workspace tools to a traditional exploratory programming system and to semantic technologies.
The semantic exploration kernel consists of two central components: a \emph{suggestion engine} for processing context from the programming system and developing suggestions, and an \emph{exploratory programming agent} for natural-language communication, machine reasoning, and autonomous experimentation.
In the following, we elaborate on our architecture of the semantic exploration kernel and describe the general operating principles of the suggestion engine and the exploratory programming agent.

\thimport{01_architecture}
\thimport{02_suggestions}
\thimport{03_agent}

\begin{summary}
	We have presented the design and architecture of the \emph{semantic exploration kernel}, which powers the semantic workspace by providing different functionalities based on semantic technologies.
	We have described the \emph{suggestion engine}, which uses a \emph{blackboard framework} for organizing suggestion artifacts and strategies, and the \emph{exploratory programming agent}, which provides \emph{semantic object interfaces} for asking semantic questions and autonomously reasons and experiments to answer these questions.
\end{summary}
