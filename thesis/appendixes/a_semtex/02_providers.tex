% !TEX root = ../../../semexp-thesis.tex

\section{Language Providers}
\label{sec:semtex/providers}

\semtex defines interfaces for text embedding and generation models that can be implemented by different \emph{providers}.
As the default provider, we implemented a basic client for the OpenAI API%
\footnote{
	\emph{OpenAI API}.
	URL:
	%\url{https://platform.openai.com/docs}
	\url{http://archive.today/2024.06.30-211140/https://platform.openai.com/docs/overview}
}.
The client accesses the HTTP endpoints using Squeak's \code{WebClient} package and supports token-based authentication and rate limits.

To support receiving generated text chunks incrementally and avoid delays, the client can handle \emph{server-sent events}\footnote{%
	``Server-Sent Events''.
	In: \emph{HTML Living Standard}, section 9.2.
	WHATWG, 2024-06-26.
	URL:
	%\url{https://html.spec.whatwg.org/commit-snapshots/\seqsplit{25aaad7f6a10785efe041fb05a597400e700ef10}/\#server-sent-events}%
	\href{https://html.spec.whatwg.org/commit-snapshots/25aaad7f6a10785efe041fb05a597400e700ef10/\#server-sent-events}{https://html.spec.whatwg.org/commit-snapshots/\seqsplit{25aaad7f6a10785efe041fb05a597400e700ef10}/\#server-sent-events}%
	.
} emitted by the API, transform them into an internal \code{Generator}, and update the returned \code{SemanticMessage} asynchronously for each new chunk.
Clients that request streamed completions initially receive an empty \code{SemanticMessage} and can subscribe to observer signals for incorporating each appended chunk into their data model or interface (such as a graphical display).

Next to the OpenAI provider, \semtex also implements a \emph{mocking provider} for simulating real language models~(see \cref{sec:semtex/tools/mocking}).
