% !TEX root = ../../semexp-thesis.tex

\chapter{Accessing Semantic Technologies through the \semtex Framework}
\label{apx:semtex}

In this appendix, we describe our design and implementation of the \semtex framework for accessing semantic technologies in Squeak.
The \semtex framework provides an object-oriented domain model for semantic search using document embeddings and for text generation and machine reasoning using conversational agents\footnote{\url{https://github.com/LinqLover/Squeak-SemanticText}}.
It also implements a client for relevant language models of the OpenAI API.
Finally, it offers basic tooling to support prototyping of semantic applications.

In the following, we describe the three central parts of the framework: domain model, providers for different language models, and tool support.

\thimport{01_model}
\thimport{02_providers}
\thimport{03_tools}

\begin{summary}
	We have described the \semtex framework for accessing semantic technologies in Squeak.
	\semtex models a \emph{semantic corpus} that can emit (ranked) \emph{search results} for embedding searches and a \emph{conversation} with \emph{messages} and \emph{functions} to implement conversational \emph{agents}.
	Searches and conversations can be powered using different \emph{language model providers} such as the built-in OpenAI API client.
	To facilitate prototyping of semantic applications despite the nondeterminism, latencies, and cost of LLMs, \semtex offers different tools for editing conversations, mocking LLMs, and estimating or tracking expenses.
\end{summary}
