% !TEX root = semexp-thesis.tex

\begin{abstract}
	In the exploratory programming practice, programmers iteratively ask questions and run experiments to understand and develop software systems.
	However, traditional exploratory programming workflows often lead to distractions and information overload, as programmers need to handle numerous implementation artifacts. % due to handling numerous implementation artifacts.
	Meanwhile, semantic technologies---text generation using large language models (LLMs) and semantic retrieval using embeddings---are establishing themselves in other development practices to assist in writing and searching code.

	We propose an augmented exploratory programming workflow that integrates semantic technologies into programming systems, allowing programmers to interact with them through more conceptual interfaces.
	Our semantic workspace introduces three semantic programming tools for augmenting and automating exploration:
	semantic suggestions anticipate the intentions of programmers and recommend possible experiments, semantic completions continue their plans through contextualized suggestions, and semantic conversations enable high-level, natural-language questions about objects.

	Our semantic exploration kernel uses semantic technologies to power these tools with a suggestion engine for recommending and contextualizing artifacts, and with an exploratory programming agent for autonomous experiments and conversations with programmers.
	We explore embedding- and term-based strategies for searching and ranking artifacts, and we design prompts and system interfaces for an agent based on the \gptfouro model.

	We successfully used a prototype of the semantic workspace for Squeak"/Smalltalk to augment different exploratory activities.
	From our experience, semantic tools show promise in streamlining the exploratory programming workflow, but they must be further optimized to master exploratory practices and semantic understanding and to reduce time and resource consumption.
	We believe that our work is an important step toward a new era of semantic exploratory programming, where programmers and deeply intelligent agents collaborate effectively to comprehend and extend large systems.
\end{abstract}

% TODO latex: only use english name in pdfbookmark
\renewcaptionname{ngerman}{\abstractname}{Zusammenfassung (German Abstract)}
\begin{zusammenfassung}
	\enlargethispage{2\baselineskip} % thank you German language
	In der Praktik des explorativen Programmierens stellen Programmierer*innen iterativ Fragen und führen Experimente durch, um Softwaresysteme zu verstehen und zu entwickeln.
	Traditionelle Workflows im explorativen Programmieren führen jedoch oft zu Ablenkungen und Informationsüberflutung, da Programmierer zahlreiche Implementierungsartefakte berücksichtigen müssen.
	Gleichzeitig etablieren sich semantische Technologien -- Textgenerierung durch große Sprachmodelle (LLMs) und semantische Suche mit Embeddings -- in anderen Bereichen der Softwareentwicklung, um das Schreiben und Suchen von Code zu unterstützen.

	Wir schlagen einen erweiterten Workflow für das explorative Programmieren vor, der semantische Technologien in Programmiersysteme integriert und es Programmierenden ermöglicht, über konzeptuelle Schnittstellen mit diesen zu interagieren.
	Unser semantischer Workspace stellt drei semantische Werkzeuge zur Verfügung, um Explorationen zu augmentieren und zu automatisieren:
	Semantische Vorschläge antizipieren die Intentionen von Programmiererinnen und empfehlen mögliche Experimente, semantische Vervollständigungen setzen ihre Pläne durch kontextualisierte Vorschläge fort, und semantische Konversationen erlauben abstrakte, natürlichsprachliche Fragen über Objekte.

	Unser semantischer Explorationskernel nutzt semantische Technologien, um diese Werkzeuge mit einem Vorschlagsmodul für Artefakte sowie einem explorativen Programmieragenten für autonome Experimente und Konversationen mit Programmierern anzutreiben.
	Wir untersuchen embedding- und termbasierte Strategien für das Suchen und Ranking von Artefakten und entwickeln Prompts und Systemschnittstellen für einen Agenten auf Basis des \gptfouro-Modells.

	Wir haben einen Prototypen des semantischen Workspaces für Squeak"/Smalltalk erfolgreich genutzt, um verschiedene explorative Aktivitäten zu augmentieren.
	Unsere Erfahrung zeigt, dass semantische Werkzeuge vielversprechend sind, um den Workflow des explorativen Programmierens zu vereinfachen.
	Sie müssen jedoch weiter optimiert werden, um explorative Praktiken und semantisches Verständnis besser zu beherrschen und den Zeit- und Ressourcenverbrauch zu reduzieren.
	Wir glauben, dass unsere Arbeit einen wichtigen Schritt in Richtung einer neuen Ära des semantischen explorativen Programmierens darstellt, in der Programmierende und tiefgreifend intelligente Agenten effektiv zusammenarbeiten, um große Systeme zu verstehen und zu erweitern.
\end{zusammenfassung}
